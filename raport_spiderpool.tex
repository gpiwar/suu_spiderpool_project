\documentclass[onecolumn,12pt]{article}

\usepackage[utf8]{inputenc}
\usepackage[T1]{fontenc}
\usepackage{polski}

\usepackage{hyperref}
\hypersetup{
    colorlinks=false, %set true if you want colored link
    linktoc=all,     %set to all if you want both sections and subsections linked
}
\urlstyle{same}

\begin{document}

% ----------Strona tytułowa------------
\begin{titlepage}
\begin{center}
\vspace*{2.5cm}
\Huge
\textbf{Spiderpool}
            
\vspace{0.5cm}
\LARGE
RDMA network solution for the Kubernetes
            
\vspace{1.5cm}

\large
Piotr Czarnik, Bartosz Kucharz
\\Gabriela Piwar, Wojciech Szmelich
              
\vspace{0.8cm}          
\Large
AGH Wydział Informatyki\\
2024    
\end{center}
\end{titlepage}

% ----------Spis treści------------
\tableofcontents
\thispagestyle{empty}
\newpage

% ----------Raport------------
\section{Wprowadzenie}
Kubernetes to jedno z najpopularniejszych narzędzi do zarządzania aplikacjami kontenerowymi. 
Z tego też względu nieustanie wprowadzane są nowe moduły usprawniające jego działanie. 

Jednym z nich jest Spiderpool - zaawansowane rozwiązanie zarządzania adresami IP (IPAM - IP Address Management) wykorzystujące technologię RDMA (Remote Direct Memory Access).
Rozszerza on standardowe interfejsy sieciowe kontenerów (CNI - Container Network Interface) umożliwiając tworzenie interfejsów Macvlan, Ipvlan, oraz SR-IOV.
Dzięki temu pozwala na większą dowolność w przypisywaniu adresów IP do kontenerów i w wykorzystaniu interfejsów sieciowych.
Natomiast SR-IOV umożliwia kontenerowi na bezpośredni dostęp do fizycznego interfejsu sieciowego - szybszy transfer danych między węzłami w klastrze Kubernetesa, minimalizacja opóźnienia i obciążenia procesora. 
Jest to szczególnie korzystne dla aplikacji wymagających wysokiej przepustowości i niskiego opóźnienia, 
jak aplikacje do przetwarzania dużych ilości danych, middleware, CNF (Container Network Functions) czy systemy baz danych.

\section{Opis technologii}

\subsection{Kubernetes}
Spiderpool działa na klastrach, czyli zestawie maszyn (węzłów) do uruchamiania skonteneryzowanych aplikacji. Kubernetes jest platformą open source do zarządzania takimi klastrami. Służy do zarządzania zadaniami i serwisami uruchamianymi w kontenerach, oraz umożliwia deklaratywną konfigurację i automatyzację. Najmniejsza i najprostsza jednostka w środowisku Kubernetes to pod, czyli grupa jednego lub wielu kontenerów aplikacji. 
W ''czystym'' k8s kontenery wewnątrz poda współdzielą adres IP i przestrzeń portów, zawsze są uruchamiane wspólnie w tej samej lokalizacji i współdzielą kontekst wykonawczy na tym samym węźle.

\subsection{AWS}
Spiderpool jest stworzone z myślą o działaniu na dowolnym środowisku chmurowym. Ułatwia również zarządzanie takimi rozwiązaniami jak multicloud czy chmura hybrydowa.\\
Jedną z najbardzej znanych i używanych platform chmurowych jest Amazon Web Services (AWS), która zapewnia szeroki wybór usług oraz zasobów obliczeniowych, sieciowych i przechowywania danych. Usługi Amazona są znacznie  rozbudowane i umożliwiają skonfigurowanie środowiska w taki sposób, aby było jak najbardziej dopasowane do danych potrzeb. Jedną z najważniejszych usług dostępnych w AWS jest Elastic Compute Cloud (EC2), która umożliwia elastyczne skalowanie zasobów obliczeniowych. Szerokie zastosowanie tej platformy oznacza również, że istnieje ogromna ilość informacji, dokumentacji i pomocy dostępnych dla użytkowników. Dlatego zdecydowano się wdrożyć projekt na tym środowisku.

\subsection{Ansible}
Ansible jest silnikiem orkiestracji pozwalającym na tworzenie oprogramowania w paradygmacie ,,infrastructure as a code''.
Umożliwia automatyzację provisioningu, konfiguracji i deploymentu systemów oraz oprogramowania za pomocą Playbook-ów - zestawów tasków, które mają się wykonać na wcześniej zdefiniowanych node'ach.

Ansible udostępnia moduły dedykowane do konfiguracji AWSa i Kubernetesa.
Wartą uwagi cechą playbooków jest idempotentność operacji - taski sprawdzają, czy dane zadanie już nie zostało wykonane, a jeśli tak to go nie powtarzają - umożliwia to wielokrotne ich uruchamianie bez konieczności czyszczenia całego środowiska.

\section{Case Study - opis projektu}
Za pomocą wymienionych uprzednio technologii, dążymy do uzyskania automatycznego deploymentu środowiska Spiderpool na chmurze AWS za pomocą Ansible'a.
W miejscach, gdzie użycie natywnych Ansible'owych modułów nie będzie możliwe, wykorzystane zostaną skrypty bash-owe lub (co jest bardziej preferowane) skrypt bash-owy wewnątrz playbooka.

Po deploymencie Spiderpoola konieczne jest zautomatyzowanie sprawdzenia poprawności działania środowiska:
\begin{enumerate}
    \item test łączności sieciowej między kontenerami - za pomocą ping-ów lub curl-ów,
    \item test działania funkcjonalności Spiderpoola - dodawanie i uwalnianie adresów IP,
    \item uruchomienie kilku przykładowych aplikacji web-wych w kontenerach używających różnych adresów IP za pomocą Macvlan lub Ipvlan.
\end{enumerate}

W ramach projektu zautomatyzowane zostaną więc:
\begin{enumerate}
    \item konfiguracja AWS - VPC i EC2,
    \item deployment K8s,
    \item deployment Spiderpoola,
    \item testy, a w tym deployment przykładowych aplikacji.
\end{enumerate}

\section{Architektura rozwiązania}
Spiderpool składa się z następujących komponentów:
\begin{enumerate}
    \item Kontroler Spiderpoola, odpowiedzialny za interakcje z API Server. 
    Zarządza różnymi zasobami CRD; SpiderIPPool, SpiderSubnet, SpiderMultusConfig; poprzez ich walidację, tworzenie i update-owanie ich statusu. 
    Dodatkowo odpowiada na żądania od Spiderpool-agent Pods, takie jak alokacja, zwolnienie czy zarządzanie pulą adresów IP. 
    \item Spiderpool-agent, zestaw daemon-ów działających na każdym nodzie. 
    Wspiera instalacje wtyczek takich jak Multus, Coordinator, IPAM, czy CNI. 
    Odpowiada na żądania alokacji IP przez CNI, podczas tworzenia Pod-u. 
    Komunikuje się z Spiderpool-controller w zakresie alokacji i tworzenia IP Pod-u. 
    Komunikuje się także z Coordinator w zakresie synchronizacji konfiguracji i wspierania implementacji alokacji IP.
    \item Wtyczek CNI
    \newline Spiderpool IPAM plugin, główny CNI używany do zarządzania alokacją adresów IP.
    \newline Coordinator plugin, odpowiada za koordynację routingu, połączenie pomiędzy hostami. 
    Zapewnia unikalność adresów IP, odpowiednią adresacje MAC.
    \newline Ifacer plugin, automatyzuje tworzenie wirtualnych interfejsów dla VLAN-ów i Bondów (łączenia wielu interfejsów sieciowych w jedno logiczne połączenie).
    \newline Multus CNI, zarządza (scheduler) innymi wtyczkami CNI.
    \item Komponentów SR-IOV
    \item Komponentów RDMA
    \newline RDMA CNI, implementuje izolację sieci
\end{enumerate}


\section{Konfiguracja środowiska}

W celu realizacji założeń projektowych związanych z automatycznym deploymentem Spiderpoola 
przeprowadzono konfigurację środowiska w AWS. Umożliwił to dostęp do platformy AWS Academy Learner Lab 
gdzie w ramach projektu otrzymaliśmy 100 dolarów w zasobach do wykorzystania na platformie. 


Konfigurację rozpoczęto od utworzenia  ...  za pomocą konsoli AWS. 

Ostatnim etapem było postawienie klastra Kubernetes w serwisie EKS (ang. Elastic Kubernetes Service). 
Przeprowadzono to według schematu ... 
 

Dodać opis z tej stronki:

\url{https://docs.daocloud.io/en/network/modules/spiderpool/public-cloud/awscloud.html}

Ostatnim krokiem było podłączenie konta w AWS do środowiska na maszynie lokalnej. Konieczna do tego jest 
instalacja narzędzia \textit{aws-cli}. Następnie wystarczy skopiować plik \textit{credentials}, dostępny w  sekcji AWS Details w Learner Lab do folderu /.aws utworzonego po uruchomieniu komendy \textit{aws configure}.


\section{Instalacja}

Konieczne do zainstalowania:

\begin{enumerate}
    \item aws-cli - https://docs.aws.amazon.com/cli/latest/userguide/install-cliv2-linux.html ,
    \item kubectl - https://kubernetes.io/docs/tasks/tools/ ,
    \item helm -  https://helm.sh/docs/intro/install/ ,
\end{enumerate}


\section{How to reproduce - steps}

\subsection{Infrastructure as Code approach}

\section{Demo deployment steps}

\subsection{Configuration set-up}
\subsection{Data preparation}
\subsection{Execution procedure}
\subsection{Results presentation}

\section{Podsumowanie}

%\bibliography{}

\end{document}