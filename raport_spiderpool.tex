\documentclass[onecolumn,12pt]{article}

\usepackage[utf8]{inputenc}
\usepackage[T1]{fontenc}
\usepackage{polski}

\usepackage{hyperref}
\hypersetup{
    colorlinks=false, %set true if you want colored link
    linktoc=all,     %set to all if you want both sections and subsections linked
}

\begin{document}

% ----------Strona tytułowa------------
\begin{titlepage}
\begin{center}
\vspace*{2.5cm}
\Huge
\textbf{Spiderpool}
            
\vspace{0.5cm}
\LARGE
RDMA network solution for the Kubernetes
            
\vspace{1.5cm}

\large
Piotr Czarnik, Bartosz Kucharz
\\Gabriela Piwar, Wojciech Szmelich
              
\vspace{0.8cm}          
\Large
AGH Wydział Informatyki\\
2024    
\end{center}
\end{titlepage}

% ----------Spis treści------------
\tableofcontents
\thispagestyle{empty}
\newpage

% ----------Raport------------
\section{Wprowadzenie}
Kubernetes to jedno z najpopularniejszych narzędzi do zarządzania aplikacjami kontenerowymi. 
Z tego też względu nieustanie wprowadzane są nowe moduły usprawniające jego działanie. 
Jednym z nich jest Spiderpool - zaawansowane rozwiązanie zarządzania adresami IP (IPAM) wykorzystujące technologię RDMA.
Rozszerza on możliwości standardowych interfejsów sieciowych kontenerów (CNI), takich jak Macvlan, Ipvlan, oraz SR-IOV, 
dzięki czemu zapewnia  szybszy transfer danych między węzłami w klastrze Kubernetes, minimalizując opóźnienia i obciążenie procesora. 
Jest to szczególnie korzystne dla aplikacji wymagających wysokiej przepustowości i niskiego opóźnienia, 
jak aplikacje do przetwarzania dużych ilości danych, middleware, czy systemy baz danych.

\section{Opis technologii}
\subsection{Kubernetes}
Spiderpool działa na klastrach, czyli zestawie maszyn (węzłów) do uruchamiania skonteneryzowanych aplikacji. Kubernetes jest platformą open source do zarządzania takimi klastrami. Służy do zarządzania zadaniami i serwisami uruchamianymi w kontenerach, oraz umożliwia deklaratywną konfigurację i automatyzację. Najmniejsza i najprostsza jednostka w środowisku Kubernetes to pod, czyli grupa jednego lub wielu kontenerów aplikacji. Kontenery wewnątrz poda współdzielą adres IP i przestrzeń portów, zawsze są uruchamiane wspólnie w tej samej lokalizacji i współdzielą kontekst wykonawczy na tym samym węźle.
\subsection{AWS}
Spiderpool jest stworzone z myślą o działaniu na dowolnym środowisku chmurowym. Ułatwia również zarządzanie takimi rozwiązaniami jak multicloud czy chmura hybrydowa.\\
Jedną z najbardzej znanych i używanych platform chmurowych jest Amazon Web Services (AWS), która zapewnia szeroki wybór usług oraz zasobów obliczeniowych, sieciowych i przechowywania danych. Usługi Amazona są znacznie  rozbudowane i umożliwiają skonfigurowanie środowiska w taki sposób, aby było jak najbardziej dopasowane do danych potrzeb. Jedną z najważniejszych usług dostępnych w AWS jest Elastic Compute Cloud (EC2), która umożliwia elastyczne skalowanie zasobów obliczeniowych. Szerokie zastosowanie tej platformy oznacza również, że istnieje ogromna ilość informacji, dokumentacji i pomocy dostępnych dla użytkowników. Dlatego zdecydowano się wdrożyć projekt na tym środowisku.
\subsection{Ansible}

\section{Case Study - opis projektu}
Za pomocą wymienionych uprzednio technologi, dążymy do uzyskania automatycznego deploymentu aplikacji. Zatem jednym z wyznaczonych celów jest możliwie maksymalnie zautomatyzowanie deploymentu; głównie poprzez użycie ansibla. \newline 
Kolejnym celem jest sprawdzenie i kontrola poprawności działania aplikacji. Dlatego planujemy zaimplementować testy akceptacyjne. 

\section{Architektura rozwiązania}

\section{Konfiguracja środowiska}

\section{Instalacja}

\section{How to reproduce - steps}

\subsection{Infrastructure as Code approach}

\section{Demo deployment steps}

\subsection{Configuration set-up}
\subsection{Data preparation}
\subsection{Execution procedure}
\subsection{Results presentation}

\section{Podsumowanie}

%\bibliography{}

\end{document}