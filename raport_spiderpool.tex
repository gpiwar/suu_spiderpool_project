\documentclass[onecolumn,12pt]{article}

\usepackage[utf8]{inputenc}
\usepackage[T1]{fontenc}
\usepackage{polski}

% ----------Strona tytułowa------------
\begin{titlepage}
\begin{center}
\vspace*{2.5cm}
\Huge
\textbf{Spiderpool}
            
\vspace{0.5cm}
\LARGE
RDMA network solution for the Kubernetes
            
\vspace{1.5cm}

\textbf{Autorzy}
\large
\\Piotr Czarnik, Bartosz Kucharz
\\Gabriela Piwar, Wojciech Szmelich
              
\vspace{0.8cm}          
\Large
AGH Wydział Infromatyki\\
2024    
\end{center}
\end{titlepage}

% ----------Spis treści------------
\begin{document}
\tableofcontents
\thispagestyle{empty}
\newpage

% ----------Raport------------
\section{Wprowadzenie}
Kubernetes to jedno z najpopularniejszych narzędzi do zarządzania aplikacjami kontenerowymi. Z tego też względu nieustanie wprowadzane są nowe moduły usprawniające jego działanie. Jednym z nich jest Spiderpool - zaawansowane rozwiązanie zarządzania adresami IP (IPAM) wykorzystujące technologię RDMA. Rozszerza on możliwości standardowych interfejsów sieciowych kontenerów (CNI), takich jak Macvlan, Ipvlan, oraz SR-IOV, dzięki czemu zapewnia  szybszy transfer danych między węzłami w klastrze Kubernetes, minimalizując opóźnienia i obciążenie procesora. Jest to szczególnie korzystne dla aplikacji wymagających wysokiej przepustowości i niskiego opóźnienia, jak aplikacje do przetwarzania dużych ilości danych, middleware, czy systemy baz danych.

\section{Opis technologii}

\section{Case Study - założenia projektu}

\section{Architektura rozwiązania}

\section{Konfiguracja środowiska}

\section{Instalacja}

\section{How to reproduce - steps}

\subsection{Infrastructure as Code approach}

\section{Demo deployment steps}

\subsection{Configuration set-up}
\subsection{Data preparation}
\subsection{Execution procedure}
\subsection{Results presentation}

\section{Podsumowanie}

\bibliography{}

\end{document}